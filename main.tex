\documentclass{article}
\usepackage[utf8]{inputenc}
\usepackage{hyperref}
\usepackage{minted}
\usepackage{xcolor}
\usepackage[a4paper]{geometry}
\usepackage{amsmath}

\title{BUSA8090 - Assignment 1, Task 3}
\author{Samuel Gardiner (44952619)}
\date{16 April 2020}

% Setup:
\setcounter{secnumdepth}{0}

% Code highlighting:

\definecolor{mannibg}{HTML}{f0f3f3}

\newminted[bashcode]{bash}{
    style = manni,
    bgcolor = mannibg
}

\newminted[bashinline]{text}{
    style = manni,
    bgcolor = mannibg,
    breaklines,
    mathescape
}

\newminted[longinline]{text}{
    style = manni,
    bgcolor = mannibg,
    mathescape
}


\newmintedfile[bashfile]{bash}{
    style = manni,
    bgcolor = mannibg,
    linenos
}

\newmintedfile[sqlfile]{sql}{
    style = manni,
    bgcolor = mannibg
}

\newmintinline[bashsnippet]{bash}{
    style = manni,
    bgcolor = mannibg
}

\newmintinline[tttsnippet]{text}{
    bgcolor = mannibg
}

% Relational algebra macros:
% --------------------------

\newcommand{\select}[1]{
\boldsymbol{\sigma}_{#1}
}

\newcommand{\project}[1]{
\boldsymbol{\pi}_{#1}
}

\newcommand{\join}[1]{
\boldsymbol{\bowtie}_{#1}
}

% Content
% -------

\begin{document}

\maketitle

\section{Note to the marker:}

Each SQL answer is available as a \texttt{.sql} file which contains only the SQL query, and as a \texttt{.sh} file which (when executed) will print the query to the terminal, connect to the \texttt{compbiol} database on your AWS instance, and then execute the query.

All of the answer SQL queries and \texttt{bash} scripts are contained within a GitHub repository at \url{https://github.com/gardiners/A1T3}

To download all of the files and get them ready to mark, run the following on your Ubuntu AWS instance: 

\begin{longinline}
source <(curl -s https://raw.githubusercontent.com/gardiners/A1T3/master/install.sh)
\end{longinline}

Which will clone the GitHub repository into a new directory called \texttt{gardiners-A1T3}, make all of the scripts executable, and then put you into that directory ready to mark.



\section{Question 1}

\subsection{a.}

Since we already have the \texttt{expression} table in our MySQL \texttt{compbiol} database, we can export it to a file. From \texttt{man mysql} (at line 60 of the man page):

\begin{bashinline}
   ·   --batch, -B

           Print results using tab as the column separator, with each row on a new line. With this option, mysql does not use the history file.
\end{bashinline}

So, with the \texttt{--batch} argument and the file redirect operator \texttt{>}, we can write a tab-separated file from the command-line:

\begin{bashcode}
# Create a directory to store the TSV file:
mkdir -p ~/busa/A1T3
# Write the TSV file:
mysql --batch -u awkologist -p compbiol -e \
"SELECT * FROM expression \
ORDER BY gene;" > ~/busa/A1T3/expression.tsv
\end{bashcode}

We have ordered the output by \texttt{gene} to make life easier when using the \texttt{join} command from \texttt{bash} in part \textbf{c.}, below.

The tab-separated file is available at \url{https://raw.githubusercontent.com/gardiners/A1T3/master/expression.tsv} for marking.

\subsection{b.}

Using the same logic as in \textbf{a.}, above, we query for every row in the \texttt{annotation} table and then save it to a file \texttt{annotation.tsv} using the \texttt{--batch} argument to \texttt{mysql}.

\begin{bashcode}
mysql --batch -u awkologist -p compbiol -e \
"SELECT * FROM annotation \
ORDER BY gene;" > ~/busa/A1T3/annotation.tsv
\end{bashcode}

Again, we have ordered by \texttt{gene} to make part \textbf{c.}, below, easier. The tab-separated file is available at \url{https://raw.githubusercontent.com/gardiners/A1T3/master/annotation.tsv} for marking.

\subsection{c.}

Since \texttt{join} requires its input files to be sorted on the key field, both TSV files exported above were sorted on \texttt{gene} at the time of export, ensuring that the list of genes is in the same order in both files. We could also use \texttt{sort} inside a subshell for each file when calling \texttt{join}, but that would make the \texttt{join} call harder to interpret (and harder to mark).

So, assuming we are in a directory with our files named \texttt{annotation.tsv} and \texttt{expression.tsv}, we can implement the given relational algebra statement with the following \texttt{bash} command:

\begin{bashinline}
join -t $'\t' --header annotation.tsv expression.tsv | cut -f 1,2,4 > join.tsv  
\end{bashinline}

Examining each element of the line, we have:

\begin{itemize}
    \item the \bashsnippet{join} command. Since the default is to join on the first field in each input, and our files already have the key field \texttt{gene} as the first field, we do not need to explicity specify the key.
    \begin{itemize}
        \item the \bashsnippet{-t $'\t'} argument, which specifies which character should be used to separate values in the input files. \bashsnippet{$'\t'} indicates that a literal tab character should be used as the separator (the \bashsnippet{$'...'} construct is ANSI C quoting of literals). By default \bashsnippet{join} uses a space \textit{or} tab character as its delimiter, but returns space-delimited output. Since we want to pass the output to \bashsnippet{cut}, we force \bashsnippet{join} to use tab as its input and output delimiter.
        \item \bashsnippet{--header} indicates that the first line of the input files should not be matched upon, and should simply be combined and returned as the first line of the output.
        \item \bashsnippet{annotation.tsv expression.tsv} specifies the input files.
    \end{itemize}
    \item the pipe \bashsnippet{|} to send the output of \bashsnippet{join} to the input of \bashsnippet{cut}
    \item the \bashsnippet{cut} command, to select the fields that we wish to return. By default, \bashsnippet{cut} uses the tab character as its field delimeter.
    \begin{itemize}
        \item the \tttsnippet{ -f 1,2,4} argument to return the fields \texttt{gene}, \texttt{function} and \texttt{expr\_value} (the first, second and fourth fields in the joined dataset).
    \end{itemize}
    \item the file redirect operator \bashsnippet{>} to save the output to the file \texttt{join.tsv}
\end{itemize}

The answer command has been saved with the filename \texttt{join.sh} in the GitHub repository for A1T3.
\begin{itemize}
    \item The command for marking is at \url{https://raw.githubusercontent.com/gardiners/A1T3/master/join.sh}
    \item Example output is at \url{https://raw.githubusercontent.com/gardiners/A1T3/master/join.tsv}
\end{itemize}

\subsection{d.}

\subsubsection{(i)}

To find the duplicated metabolisms in the \texttt{annotation} table, our algebraic operations are
\begin{itemize}
    \item find the inner join of \texttt{annotation} with itself, with the joining conditions
    \begin{itemize}
        \item both `copies' of \texttt{metabolism} in the joint tuple are the same, AND
        \item each `copy' of \texttt{gene} in in the joint tuple are different
    \end{itemize}
    \item project the resulting list of metabolisms.
\end{itemize}

In relational algebra, our statement is therefore:

$$
\begin{aligned}
\boldsymbol{\pi}_\mathbf{metabolism}
\left(
    \mathbf{annotation} \bowtie_{\mathbf{metabolism} = \mathbf{metabolism} \text{ AND }
        \mathbf{gene} \ne \mathbf{gene}} \mathbf{annotation}
\right)
\end{aligned}
$$

\subsubsection{(ii)}

Since the relational algebra statement above is guaranteed to return distinct tuples (by its axioms), we use \texttt{SELECT DISTINCT} in our SQL query to ensure that we obtain only distinct relations as its result. That is, we wish to know only the names of the duplicated metabolisms; we don't want a repeated row for each time a metabolism is duplicated.

Therefore, our SQL query is:

\sqlfile{q1dii.sql}

which returns the result

\begin{bashinline}
+-------------------------+
| metabolism              |
+-------------------------+
| Pyrimidine biosynthesis |
+-------------------------+ 
\end{bashinline}

\begin{itemize}
    \item an SQL command for marking is available at \url{https://raw.githubusercontent.com/gardiners/A1T3/master/q1dii.sql}
    \item an executable \texttt{bash} script that runs the SQL command at the terminal is available at \url{https://raw.githubusercontent.com/gardiners/A1T3/master/q1dii.sh}
\end{itemize}

\section{Question 2}

\subsection{a.}

\subsubsection{(i)}

Assuming our dialect of relational algebra has a definition for NULL,

$$
\project{\mathbf{LastName, FirstName}}
\left(
\select{\mathbf{Coach} \text{ IS NULL}}\left(\mathbf{Member}\right)
\right)
$$

\subsubsection{(ii)}

Assuming our dialect of the relational calculus has a definition for NULL,

$$
\left\{
\texttt{ m.LastName, m.FirstName | Member(m) and m.Coach IS NULL }
\right\}
$$

\subsubsection{(iii)}

Our query is

\sqlfile{q2a.sql}

which returns

\begin{bashinline}
+----------+-----------+
| LastName | FirstName |
+----------+-----------+
| Stone    | Michael   |
| Nolan    | Brenda    |
| Branch   | Helen     |
| Beck     | Sarah     |
| Spence   | Thomas    |
| Olson    | Barbara   |
| Wilcox   | Daniel    |
| Young    | Betty     |
| Willis   | Carolyn   |
| Kent     | Susan     |
+----------+-----------+
\end{bashinline}


\begin{itemize}
    \item The SQL query for marking is available at \url{https://raw.githubusercontent.com/gardiners/A1T3/master/q2a.sql}
    \item An executable \texttt{bash} script that will run the query against the \texttt{compbiol} database is available at \url{https://raw.githubusercontent.com/gardiners/A1T3/master/q2a.sh}
\end{itemize}

\subsection{b.}

\subsubsection{(i)}

We select from Member all of the players who meet the condition that their join date is between the two dates specified, and then project the atributes LastName and FirstName:

$$
\project{\mathbf{LastName, FirstName}}
\left(
\select{\mathbf{JoinDate} \geq \text{`2010-01-01' AND } \mathbf{JoinDate \leq \text{`2010-12-31'}}}
\left(\mathbf{Member}\right)
\right)
$$

\subsubsection{(ii)}

$$
\begin{aligned}
\{
\texttt{ m.LastName, m.FirstName } | &\texttt{ Member(m)} \\
    &\texttt{ and m.JoinDate} \geq \texttt{`2010-01-01'} \\
    &\texttt{ and m.JoinDate} \leq \texttt{`2010-12-31' }
\}
\end{aligned}
$$

\subsubsection{(iii)}

We can use \texttt{BETWEEN} as a syntactic sweetener instead of a pair of comparisons. A query that implements the above expressions (when executed against \texttt{compbiol}) is 

\sqlfile{q2b.sql}

which returns:

\begin{bashinline}
+----------+-----------+
| LastName | Firstname |
+----------+-----------+
| Beck     | Sarah     |
| Kent     | Susan     |
+----------+-----------+
\end{bashinline}

\begin{itemize}
    \item The SQL for marking is at \url{https://raw.githubusercontent.com/gardiners/A1T3/master/q2b.sql}
    \item An executable \texttt{bash} script that runs the SQL against \texttt{compbiol} is available at \url{https://raw.githubusercontent.com/gardiners/A1T3/master/q2b.sh}
\end{itemize}

\subsection{c.}

\subsubsection{(i)}

In the algebraic paradigm, we perform two subequeries to assumble the filter for the select operator:
$$
\begin{aligned}
\mathbf{m1} &\leftarrow \project{\mathbf{MemberID}}(\mathbf{Entry})\\
\mathbf{m2} &\leftarrow \project{\mathbf{MemberID}}\left(\select{\mathbf{Year} = 2014} \left(\mathbf{Entry}\right) \right)\\
\end{aligned}
$$

The first of the subqueries (here \textbf{m1}) simply lists all of the MemberIDs that have competed in tournaments. The second subquery (\textbf{m2}) lists all of the MemberIDs who competed in 2014. We can use these subqueries together in our main algebraic statement to perform a set difference:

$$
\project{\mathbf{MemberID, LastName, FirstName}}
\left(
    \select{\mathbf{MemberID} \texttt{ IN } \mathbf{a1} \texttt{ AND } \mathbf{MemberID} \texttt{ NOT IN } \mathbf{a2}}
    \left(\mathbf{Member}\right)\right)
$$

\subsubsection{(ii)}

$$
\begin{aligned}
\{
\texttt{ m.MemberID, m.LastName, m.FirstName } | &\texttt{ Member(m), Entry(e)} \\
    &\texttt{ and m.MemberID} = \texttt{e.MemberID} \\
    &\texttt{ and NOT } \exists \texttt{(n) Entry(n) and n.Year} = 2014
\texttt{ } \}
\end{aligned}
$$

\subsubsection{(iii)}

A query which implements the expressions above (in the relational calculus style) is

\sqlfile{q2c.sql}

which yields the result:

\begin{bashinline}
+----------+----------+-----------+
| MemberID | LastName | FirstName |
+----------+----------+-----------+
|      228 | Burton   | Sandra    |
|      239 | Spence   | Thomas    |
+----------+----------+-----------+
\end{bashinline}

\begin{itemize}
    \item The SQL answer for marking is at \url{https://raw.githubusercontent.com/gardiners/A1T3/master/q2c.sql}
    \item A \texttt{bash} script which executes the SQL against \texttt{compbiol} is available at  \url{https://raw.githubusercontent.com/gardiners/A1T3/master/q2b.sh}
\end{itemize}

\subsection{d.}

\subsubsection{(i)}

$$
\begin{aligned}
\{
\texttt{ m.MemberID, m.LastName, m.FirstName |} &\texttt{ Member(m) AND }\\
&\forall \texttt{(y) Entry (y)}\\
&(\exists \texttt{(e) Entry(e) AND y.Year = e.Year AND}\\
&\texttt{m.MemberID = e.MemberID})
\}
\end{aligned}
$$

We can also express this as a double-negative to aid translation to SQL: we want every member where there does not exist a year in which there does not exist an entry for that player.

\subsubsection{(ii)}

We can express this without the double-negative in SQL if we take the algebraic approach. We use \texttt{COUNT(DISTINCT Years)} for our member-entries and compare this to the total number of distinct years in which members having been playing using the \texttt{HAVING} clause and a subquery:

\sqlfile{q2d.sql}

which returns only the players who played in 2013, 2014 and 2015:

\begin{bashinline}
+----------+----------+-----------+
| MemberId | LastName | FirstName |
+----------+----------+-----------+
|      235 | Cooper   | William   |
|      286 | Pollard  | Robert    |
|      415 | Taylor   | William   |
+----------+----------+-----------+
\end{bashinline}

\begin{itemize}
    \item The SQL for marking is at \url{https://raw.githubusercontent.com/gardiners/A1T3/master/q2d.sql}
    \item An executable \texttt{bash} script which prints the query text, runs the query against the \texttt{compbiol} database and then returns the results is available at \url{https://raw.githubusercontent.com/gardiners/A1T3/master/q2d.sh}
\end{itemize}

\end{document}